\documentclass[]{article}

\usepackage[table]{xcolor}
\setlength{\arrayrulewidth}{1mm}
\setlength{\tabcolsep}{18pt}
\renewcommand{\arraystretch}{1.5}

\title{Ricerca Euristica progetto esame di Intelligenza Artificiale}
\date{2018-2019}
\author{Mandelli Lorenzo}

\begin{document}

\maketitle

\begin{abstract}
	\noindent \'E stato realizzato un problema di ricerca di stato elaborando e definendo una specifica metrica euristica e implementando l'algoritmo di ricerca A*. Infine sono stati confrontati i risultati della performance ottenuti con la ricerca cieca e la ricerca informata.
\end{abstract}

\section{Assegnamento: assemblaggio di prodotti}
	L’assemblaggio di un certo prodotto richiede lo svolgimento di \textbf{N} compiti \{c1,...,cn\}, ognuno dei quali richiede un tempo (discreto) ti,i = 1,..., n per essere completato. 
	Alcuni compiti richiedono certi prerequisiti, ovvero il compito ci richiede il completamento dei compiti Pi $\subset$\{c1,...,cn\}  prima di  poter iniziare (Pi è l’insieme vuoto se non ha prerequisiti).
	Sono disponibili \textbf{M} impiegati ai quali si possono assegnare i compiti e ogni impiegato può solo svolgere un compito ad un determinato istante di tempo. 
	Supponendo di dover assemblare \textbf{K} copie identiche del prodotto, va trovata la sequenza temporale di assegnamento dei
	compiti ai dipendenti in modo da minimizzare il tempo complessivo per completare tutti gli assemblaggi.
	Si scriva un programma (in un linguaggio di programmazione a scelta) basato su tecniche di ricerca per risolvere il problema, definendo un’euristica da utilizzare con A*.

\section{Implementazione}

	Al fine di eseguire l’elaborato assegnato sono state usate le seguenti classi:
	\begin{itemize}
		\item{\textbf{Compito}}: classe che rappresenta i singoli task dei Prodotti da eseguire. A ogni
			compito è assegnato un tempo di esecuzione, un identificatore e una lista di
			riferimenti ad altri compiti dello stesso Prodotto che rappresentano i suoi requisiti.
			Un compito non può essere assegnato a un Impiegato fino a che tutti i suoi requisiti
			non sono stati completati.
		\item{\textbf{Prodotto}}: classe che racchiude un insieme di Compiti. Essi si dividono in:
			disponibili (avaibleCompiti), non disponibili (unavaibleCompiti), ultimati
			(doneCompiti) e in lavorazione (compitiWorkingOn). Sono disponibili inoltre i
			metodi di creazione di copia di un Prodotto e un metodo per la generazione di un
			Prodotto casuale, quest’ultimo metodo produce i requisiti tra compiti in base a una
			probabilità impostabile dall’utente.
		\item{\textbf{State}}: classe che rappresenta lo stato del problema a un certo istante di tempo. Essa
			è caratterizzata da un attributo (numberFreeEmployers) che indica il numero di
			impiegati liberi e da due liste (doneProductList, undoneProductList) di oggetti
			Prodotto che rappresentano i quali Prodotti sono stati ultimati e quali non.
		\item{\textbf{Node}}: classe che implementa a struttura dei nodi utilizzata dall’algoritmo preso in
			esame, a\_star\_search, per navigare nell’albero di ricerca. Tiene traccia dello stato
			(state), della profondità (depth), dell’azione utilizzata per giungere a se stesso
			(action), del costo necessario (path\_cost) e del nodo “padre” da cui deriva (parent).
			Fornisce inoltre i metodi per generare da un Nodo tutti i possibili Nodi “figli”.
		\item{\textbf{Problem}}: classe che definisce la struttura astratta per la classe RealProblem.
		\item{\textbf{RealProblem}}: classe che rappresenta il problema di assemblaggio di prodotti preso
			in esame. E’ caratterizzata da un riferimento allo stato iniziale (initial) e un
			riferimento allo stato goal (goal) che crea a partire da quello iniziale fornito nel
			costruttore. Implementa i metodi:
			\begin{itemize}
				\item{action}: che ritorna la lista delle possiili azioni effettuabili in un certo Stato.
				\item{result}: che ritorna lo stato risultante dalla scelta di una particolare azione in un
					particolare stato.
				\item{goal\_test}: che analizza uno Stato e ritorna True se esso coincude con lo stato goal.
				\item\textbf{{h}}: che, data uno Stato, analizza le tre euristiche h1, h2 e h3 per fare una stima della
					“distanza” che lo separa dallo Stato goal.
				\item{\textbf{h1}}: ritorna il massimo tempo di esecuzione presente tra tutti i compiti (disponibili,
					non disponibili e in lavorazione) di tutti i prodotti non ultimati.
				\item{\textbf{h2}}: ritorna il tempo massimo necessario a ultimare un compito attualmente non
					disponibile considerando la somma maggiore delle catene dei suoi requisiti.
				\item{\textbf{h3}}: ritorna la stima del tempo necessario a ultimare tutti i compiti considerando la somma dei tempi di ogni compiti diviso il numero totale di impiegati.
			\end{itemize}
		
		\section{Risultati}	
			Dai dati empirici, ottenuti utilizzando la probablità di formazione di un requisito di default (10\%), si ottengono le seguenti tabelle:
			
			\subsection{Algoritmo A\_star\_search}:
			
				\begin{table}
					\centering
					{\rowcolors{2}{green!80!yellow!50}{green!70!yellow!40}
						\begin{tabular}{ |p{0.9cm}|p{0.9cm}|p{0.9cm}|p{0.9cm}|p{0.9cm}|  }
							\hline
							M& K & N & N*M & Media tempi(s)\\
							\hline
							1 & 1  & 10 & 10 & 0.039  \\
							5 & 1  & 10 & 10 & 0.016  \\
							10 & 1  & 10 & 10 & 0.015  \\
							\hline
						\end{tabular}
					}
				\caption{\textbf{(A)}}
				\end{table}
				
				\begin{table}
					\centering
					{\rowcolors{2}{green!80!yellow!50}{green!70!yellow!40}
						\begin{tabular}{ |p{0.9cm}|p{0.9cm}|p{0.9cm}|p{0.9cm}|p{0.9cm}|  }
							\hline
							M& K & N & N*M & Media tempi (s)\\
							\hline
							1 & 2  & 10 & 20 & 0.273 \\
							10 & 2  & 10 & 20 &  0.024 \\
							20 & 2  & 10 & 20 & 0.026  \\
							\hline
						\end{tabular}
					}
				\caption{\textbf{(B)}}
			\end{table}
				
				\begin{table}
					\centering
				{\rowcolors{2}{green!80!yellow!50}{green!70!yellow!40}
					\begin{tabular}{ |p{0.9cm}|p{0.9cm}|p{0.9cm}|p{0.9cm}|p{0.9cm}|  }
						\hline
						M& K & N & N*M & Media tempi(s)\\
						\hline
						1 & 3  & 10 & 30 & 0.436 \\
						15 & 3  & 10 & 30 &  0.080 \\
						30 & 3  & 10 & 30 & 0.069  \\
						\hline
					\end{tabular}
				}
			\caption{\textbf{(C)}}
		\end{table}
			
			\begin{table}
				\centering
				{\rowcolors{2}{green!80!yellow!50}{green!70!yellow!40}
					\begin{tabular}{ |p{0.9cm}|p{0.9cm}|p{0.9cm}|p{0.9cm}|p{0.9cm}|  }
						\hline
						M& K & N & N*M & Media tempi(s)\\
						\hline
						1 & 4  & 10 & 40 & 0 1,031 \\
						20 & 4  & 10 & 40 &  0.202 \\
						40 & 4  & 10 & 40 & 0.177  \\
						\hline
					\end{tabular}
				}
			\caption{\textbf{(D)}}
		\end{table}
			
			\begin{table}
				\centering
				{\rowcolors{2}{green!80!yellow!50}{green!70!yellow!40}
					\begin{tabular}{ |p{0.9cm}|p{0.9cm}|p{0.9cm}|p{0.9cm}|p{0.9cm}|  }
						\hline
						M& K & N & N*M & Media tempi(s)\\
						\hline
						1 & 5  & 10 & 50 & 2,101 \\
						25 & 5  & 10 & 50 &  0.326 \\
						50 & 5  & 10 & 50 & 0.315  \\
						\hline
					\end{tabular}
				}
			\caption{\textbf{(E)}}
		\end{table}
		

			\subsection{Algoritmo Best\_first\_graph\_search}:
			
			\begin{table}
				\centering
				{\rowcolors{2}{green!80!yellow!50}{green!70!yellow!40}
					\begin{tabular}{ |p{0.9cm}|p{0.9cm}|p{0.9cm}|p{0.9cm}|p{0.9cm}|  }
						\hline
						M& K & N & N*M & Media tempi(s)\\
						\hline
						1 & 1  & 10 & 10 & 15,809 \\
						5 & 1  & 10 & 10 &  1.016 \\
						10 & 1  & 10 & 10 & 0.006  \\
						\hline
					\end{tabular}
				}
			\caption{\textbf{(F)}}
		\end{table}
			
			\begin{table}
				\centering
				{\rowcolors{2}{green!80!yellow!50}{green!70!yellow!40}
					\begin{tabular}{ |p{0.9cm}|p{0.9cm}|p{0.9cm}|p{0.9cm}|p{0.9cm}|  }
						\hline
						M& K & N & N*M & Media tempi(s)\\
						\hline
						1 & 1  & 15 & 15 & 0 26,101 \\
						7 & 1  & 15 & 15 &  0.028 \\
						15 & 1  & 15 & 15 & 0.021  \\
						\hline
					\end{tabular}
				}
			\caption{\textbf{(G)}}
		\end{table}	
			
			\subsection{Conclusioni}
			
				Dai dati si evince la superiorità nel tempo di secuzione di \textit{a\_star\_search} rispetto a 
				\textit{best\_first\_graph\_search} a parità di input . Le euristiche hanno quindi permesso di 
				ottenere migliori performance rispetto all’utilizzo del solo path\_cost come criterio di 
				esplorazione dell’albero.\\
				Si nota inoltre che all’aumentare del numero di impiegati diminuisce il tempo 
				necessario al completamento del problema. \\
				Gli esperimenti indicano che il tempo di esecuzione può notevolmente variare (anche
				di un fattore 100) in base all’istanza di input presente. Essendo i vincoli tra compiti 
				generati randomicamente si è ottenuto che in caso di maggior numero di vincoli 
				presenti la difficoltà del problema decresce, mentre aumenta con la maggiore 
				presenza di compiti privi di requisiti. Ciò è dovuto al diminuire di possibili strade 
				percorribili dall’algoritmo a ogni istante di tempo in caso di maggior numero di 
				vincoli.
				
	\end{itemize}
	
	
	
	
	

\end{document}
